\documentclass{article}
\usepackage{geometry}
\usepackage{graphicx}
\usepackage[pdftex, pdfborder={0 0 0}] {hyperref}
\geometry{body={6in, 10.0in},
  left=0.65in,
  top=0.35in, 
  bottom=0.25in,
right = 0.65in
}
\date{}
\begin{document}
% \thispagestyle{empty}

\section*{\huge \center{PANKAJ PRATEEK KEWALRAMANI}}
\line(1,0){520}\\ \\
\begin{minipage}{0.5\textwidth}
Third Year Dual Degree Student\\
CSE Department\\
IIT Kanpur\\
\end{minipage}
\begin{minipage}{0.5\textwidth}\begin{flushright}
Mobile: (+91) 9450039421\\
Email: pratikkr@iitk.ac.in\\
pratikkr@cse.iitk.ac.in\\
pankaj200292@gmail.com\\ \end{flushright}
\end{minipage}
\line(1,0){520}

\section*{EDUCATION}

\begin{center}
\begin{tabular}{ | c | c | c | c | c | }
\hline
\textbf{Year} & \textbf{Degree} & \textbf{Institution/Board} & \textbf{CPI / \% age} \\ \hline
2010 - 2014$^{*}$ & B.Tech & Indian Institute of Technology, Kanpur & 8.7/10 (=3.5/4) \\ \hline
2010 & XII & DAV Public School, Kota (C.B.S.E.) & 91\%  \\ \hline
2008 & X  & St. Anne's Sr. Sec. School, Jodhpur(C.B.S.E.) & 95\%  \\ \hline
\end{tabular}
\end{center}
\\
${}^*$Expected Year of Graduation

\section*{AREAS OF INTEREST}
\begin{itemize}
	\item Design and Analysis of Algorithms
	\item Artificial Intelligence, Machine Learning (Automated Systems) and Computer Vision
	\item Number Theory and Cryptology
	\item Programming Languages
	\item Compilers and Compilers Concurrency
	\item Databases and Data Mining
	\item Natural Language Processing
	\item Cognitive Science
\end{itemize}

\section*{PROJECTS}

\begin{description}

\item[SECURE IP] \hfill \\
Course Project: Computer Networks \textit{(Nov '12)}
\begin{list}{-}{}
\item Implemented a Secure IP to send secure data packets across the network
\item Used the \textbf{AES} scheme for encrypting and decrypting data packets and \textbf{md5 hash} to check the integrity of the data
\item A Graphical interface to establish peer to peer connection also facilitating data input and displaying the received data.
\end{list}

\item[EXTENSION OF PINTOS] \hfill \\
Course Project: Operating Systems \textit{(Aug '12 - Oct '12)}
\begin{list}{-}{}
\item Have to design various functionalities in PINTOS - instructional software that runs as a secondary OS on Linux.
\item Includes implementation of techniques for synchronization, demand paging, scheduling, virtual memory management, page fault and swapping, and simple system calls.
\item Total implementation in 'C'.
\end{list}

\item[PARSER GENERATOR IN PYTHON] \hfill \\
Mentor: Dr. Amey Karkare, CSE, IITK \textit{(May '12 - July '12)}
\begin{list}{-}{}
\item Built a parser generator in Python which takes a BNF form grammar file and performs lexical analysis on it.
\item To generate the set of rules and determine if the set is valid or not
\item Compute the first set, follow set, LL(1) Parse Table, LR(0) Item Sets and Parse Table, SLR(1) Parse Table, LR(1) Item Sets and Parse Table and LALR(1) Item Sets and Parse Table.
\item Generated a parser to parser grammar expressions using the above.
\item Designed the GUI using PHP and hosted the working model on the CSE local server.
\end{list}

\item[USING PSS TO SOLVE SUDOKU CSP] \hfill \\
Course Project: Artificial Intelligence Programming \textit{(Feb '12 - Apr '12)}
\begin{list}{-}{}
\item Implemented Progressive Stochastic Search (PSS) and Iterative PSS to solve Sudoku.
\item Modelled Sudoku as a Constraint Satisfaction Problem and implemented it using C++.
\end{list}

\item[MACHINE LEARNING]  \hfill \\
Summer Project under Programming Club, IIT Kanpur \textit{(May '11 - June '11)}
\begin{list}{-}{}
\item Developed a Tron bot (based on the movie Tron), incorporating in it learning procedures so that it improves it skills as it plays against other AI bots.
\item Use of pygame package of Python for developing the graphical interface of the bot.
\end{list}

\item[ARCHITECTURE PROJECT] \hfill \\
Course Project: Computer Organization and Architecture \textit{(August '11 - October '11)}
\begin{list}{-}{}
\item Programmed a 32-bit ALU on a XILINX Spartan3 FPGA board, using BSV as hardware programming language.
\item Arithmetic and logical operations could be easily done using this system.
\end{list}

\item[WEB DEVELOPMENT] \hfill \\
Yahoo HackU! 2011 \textit{(August '11)}
\begin{list}{-}{}
\item Made a portal for online quizzing. Apart from the normal quizzing modules, it would take the question from the quizmaster and grade it for difficulty by searching for the question and the answer throughout the Web.
\item This was done by integerating Yahoo APIs with PHP and cURL using OAuth.
\end{list}

\item[APPLICATION DESIGNING] \hfill \\
Code-a-Thon 2011 under ACA, IIT Kanpur \textit{(Feburary '11)}
\begin{list}{-}{}
\item Developed a peer-to-peer messenger for exchanging messages between two computers connected on a LAN network.
\item Used C for developing the messenger.
\end{list}
\end{description}

\section*{ACHIEVEMENTS}
\begin{itemize}
\item {\bf Codeforces} rating of {\bf 2500+} : Profile (handle: boygenius)
\item World rank in {\bf top 700} at {\bf Spoj} : Profile (handle: pankaj\_prateek)
\item Among the {\bf top 250} participants from India in {\bf Project Euler} (handle: pankaj\_prateek)
\item Ranked {\bf 12} (handle: boygenius) in the Inter-IIT Programming Contest, Interview Street : Rank list
\item Secured All India Rank of 157 in JEE-2010 competing against 450,000 applicants
\item Was among top 0.2\% in All India Engineering Enterance Exam, 2010 conducted by CBSE (AIR-160)
\item Awarded {\bf Gold Medal} in the {\bf Orientation-cum-Selection Camp(OCSC)} for the {\bf International Chemistry Olympiad} conducted by {\bf HBCSE, TIFR} in the year 2010
\item Qualified for {\bf Indian National Olympiad in Informatics (INOI)} in 2008
\item Awarded the prestigious {\bf Kishore Vigyan Protsahan Yojana(KVPY)} fellowship by the Department of Science and Technology, Government of India in the year 2008
% \item Awarded the {\bf National Talent Search Examination(NTSE) Scholarship} from 2008 onwards
% \item Awarded the {\bf State Talent Search Examination(SSTSE) Scholarship} for securing 9th position in SSTSE organized by the Government of Rajasthan in the year 2008
% \item Selected amongst the top 1\% students across the nation who appeared for the {\bf National Standard Examination in Astronomy} in the years 2009 and 2010
% \item Secured 2nd position in the national-level Map Quiz conducted by the {\bf Indian National Cartographic Association} in 2007
\end{itemize}

\section*{TECHNICAL SKILLS}
\begin{list}{-}{}
\item Efficient in coding in C, C++, Python
\item Assembly Language (MIPS Architecture), BSV, experience with FPGA Boards, PLY (parsing tool in python), Oz
\item Latex, Gnuplot, Octave, HTML, Javascript, CSS, PHP, Bash, Xilinx ISE, AutoCAD, Lex, Yacc, Matlab
\end{list}

% \section*{RELEVANT COURSES}
% \begin{tabular}{p{10cm} p{10cm}}
% Data Structures and Algorithms & Discrete Mathematics \\ 
% Operating Systems & Computer Networks \\
% Principles of Programming Languages & Theory of Computation \\
% Fundamentals of Computing (in C) & Computer Organisation and Architecture \\ 
% Probability and Statistics & Mathematical Logic \\
% Programming Tools and Techniques & Introduction to Electronics\\
% Differential Equations & Linear Algebra \\
% Interpersonal Dynamics & Introduction to Psychology \\
% \end{tabular} \\

\section*{MISCELLANEOUS}
\begin{itemize}
\item Coordinator of ACA (Association of Computing Activities), Student body of CSE department, IIT Kanpur (2012-2013)
\item Coordinator of IOPC (International Online Programming Contest) and Software Corner for Techkriti 2013
\item Attended a workshop on Ethical Hacking and Cyber Security by CyberCure Solutions organized in IIT Kanpur in September 11
\item Won Kodefest and came third in Semester Programming Contest organized by Programming Club, IITK
\item Worked as a Student Guide and Academic Mentor in the institute Counselling Service Team 2011-2012
\end{itemize}

\end{document}
