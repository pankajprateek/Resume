\documentclass[a4paper,10pt]{article}

%A Few Useful Packages
\usepackage{marvosym}
\usepackage{fontspec} 					%for loading fonts
\usepackage{xunicode,xltxtra,url,parskip} 	%other packages for formatting
\RequirePackage{color,graphicx}
\usepackage[usenames,dvipsnames]{xcolor}
\usepackage[big]{layaureo} 				%better formatting of the A4 page
% an alternative to Layaureo can be ** \usepackage{fullpage} **
\usepackage{supertabular} 				%for Grades
\usepackage{titlesec}					%custom \section
\usepackage{pbox}
\usepackage{fullpage}
\usepackage{longtable}

%Setup hyperref package, and colours for links
\usepackage{hyperref}
\definecolor{linkcolour}{rgb}{0,0.2,0.6}
\hypersetup{colorlinks,breaklinks,urlcolor=linkcolour, linkcolor=linkcolour}

%FONTS
\defaultfontfeatures{Mapping=tex-text}
%\setmainfont[SmallCapsFont = Fontin SmallCaps]{Fontin}
%%% modified for Karol Kozioł for ShareLaTeX use
%\setmainfont[
%SmallCapsFont = Fontin-SmallCaps.otf,

%BoldFont = Fontin-Bold.otf,

%ItalicFont = Fontin-Italic.otf
%]
%{Fontin.otf}
%%%

%CV Sections inspired by: 
%http://stefano.italians.nl/archives/26
\titleformat{\section}{\Large\scshape\raggedright}{}{0em}{}[\titlerule]
\titlespacing{\section}{0pt}{3pt}{3pt}
%Tweak a bit the top margin
%\addtolength{\voffset}{-1.3cm}

%Italian hyphenation for the word: ''corporations''
\hyphenation{im-pre-se}

%-------------WATERMARK TEST [**not part of a CV**]---------------
\usepackage[absolute]{textpos}

\setlength{\TPHorizModule}{30mm}
\setlength{\TPVertModule}{\TPHorizModule}
\textblockorigin{2mm}{0.65\paperheight}
\setlength{\parindent}{0pt}

%--------------------BEGIN DOCUMENT----------------------
\begin{document}

%WATERMARK TEST [**not part of a CV**]---------------
%\font\wm=''Baskerville:color=787878'' at 8pt
%\font\wmweb=''Baskerville:color=FF1493'' at 8pt
%{\wm 
%	\begin{textblock}{1}(0,0)
%		\rotatebox{-90}{\parbox{500mm}{
%			Typeset by Alessandro Plasmati with \XeTeX\  \today\ for 
%			{\wmweb \href{http://www.aleplasmati.comuv.com}{aleplasmati.comuv.com}}
%		}
%	}
%	\end{textblock}
%}

\newcommand{\specialcell}[2][c]{%
  \begin{tabular}[#1]{@{}c@{}}#2\end{tabular}}
  
\pagestyle{empty} % non-numbered pages

\font\fb=''[cmr10]'' %for use with \LaTeX command

%--------------------TITLE-------------
\par{\centering
		{\Huge \textsc{Pankaj Prateek Kewalramani}
	}\par}

%--------------------SECTIONS-----------------------------------
%Section: Personal Info
\section{Personal Information}

\begin{tabular}{rlrl}
    \textsc{Address:}   & B218, Hall-9, IIT Kanpur, Kanpur (India) \\
    \textsc{Phone:}     & +91 94 50 039421\\
    \textsc{email:}     & \href{mailto:pratikkr@cse.iitk.ac.in}{pratikkr@cse.iitk.ac.in},
                        \href{mailto:pankaj200292@gmail.com}{pankaj200292@gmail.com} \\
    \textsc{Web:}       & \href{http://home.iitk.ac.in/~pratikkr}{home.iitk.ac.in/\textasciitilde pratikkr}
\end{tabular}

%Section: Education first
\section{Education}
\begin{longtable}{rl}	
 2010-2015\footnotemark[1] & Master of Technology, \textbf{Indian Institute of Technology Kanpur}, India\\
& Thesis: ``Approximation Algorithms for Common Subtree and Related Problems'' \\
& \small Advisors: Prof. Shashank Kumar Mehta\\
& \normalsize Major: Computer Science and Engineering\\
& \textsc{Gpa}: 10.0/10.0\\&\\
2010-2015\footnotemark[1] & Bachelor of Technology, \normalsize\textbf{Indian Insitute of Technology Kanpur}, India\\
&\normalsize Major: Computer Science and Engineering\\
& \textsc{Gpa}: 8.8/10.0\\&\\
2008-2010 & 12\textsuperscript{th} Grade (High School), \normalsize\textbf{DAV Public School, Kota}, India\\
&\normalsize Major Subjects: Mathematics, English, Physics, Chemistry\\
& \textsc{Result}: 91\%\\
\end{longtable}
\footnotetext[1]{Expected to be completed in June 2015.}

\section{Publications}
\textsc{Anaphora without syntax - A Multi-lingual Approach for Geometry Constructions}\\
  {\small Pankaj P., Jeetesh M., Amey K., Sumit G., Amitabha M. Anaphoras without syntax - in a Geometry Construction context. To appear in ICON, 2014.}
       
%Section: Research Experience after Education
\section{Research Experience}
\begin{longtable}{r|p{12cm}}
\textsc{Jan 2014 - Present} & M.Tech. Thesis: Approximation Algorithms for common subtree \& related problems\\
&\small{Mentor: Prof. S. K. Mehta}\\
& \small{Investigating approximation algorithms and parameterization techniques for the common sub-tree problem and trying to apply them to specialized classes of graphs and related problems like tree edit distance. We are looking into semi-definite relaxations of the problem to find an approximate common subtree. Prior to that, we worked on at is counting number of paths of length less than L between two vertices in a directed graphs and on approximating the number of lines needed to cover a set of planar points (\textsc{Covering Points by Lines}).}\\

 \multicolumn{2}{c}{} \\
 \textsc{Aug 2013 - Present} & Anaphora without syntax\\
 &\small{Mentor: Prof. A. Mukerjee, Prof. A. Karkare, Dr. Sumit Gulwani (MSR Redmond))}\\
 & \small{Designed a language-independent system for high-school geometry construction problems achieveing an accuracy of more than 90\% for English and Hindi using cross lingual mapping (probabilistically mapping constructs/words in different languages), heuristic based parsing and context based semantic analysis to handle anaphora.}\\
& \small{Project Webpage: \href{http://cse.iitk.ac.in/users/pratikkr/btp/index.html}{http://cse.iitk.ac.in/users/pratikkr/btp/index.html}}\\

  \multicolumn{2}{c}{} \\
 \textsc{Summer 2014} & Hunting Compiler Concurrency bugs using x86-tso memory model \\
 & \small{Mentor: Dr. Francesco Zappa Nardelli, Team Parkas, INRIA, Paris-Rocquencourt}\\
& \small{We aimed at hunting concurrency bugs in GCC and CLANG to improve compiler optimizations for multi-processor applications by adding global memory trace to study effect of compiler optimizations on global memory accesses, and replay instrumentation to study the manner in which a load instruction affects subsequent instructions and support for control dependency analysis to study effects of conditional statements, and MMX/SSE (128 bit SIMD) instructions}\\
&\small{Project Link: \href{https://raweb.inria.fr/rapportsactivite/RA2p013/parkas/uid30.html}{https://raweb.inria.fr/rapportsactivite/RA2p013/parkas/uid30.html}}\\


\multicolumn{2}{c}{}\\
\textsc{Mar 2013 - Apr 2013} & Multi-Lingual word learning for containment situations\\
& \small{Mentor: Prof. Amitabha Mukerjee}\\
& \small{Attempted to learn synonymous words in multiple languages for a given context using common ground semantics \& label association. Learned design specifications in the given context (peg-in-a-hole) and then tried to learn linguistic semantics for them. }\\
&\small{Project Webpage: \href{http://home.iitk.ac.in/users/pratikkr/se367/project/}{http://home.iitk.ac.in/users/pratikkr/se367/project/}}\\

\end{longtable}


\section{Teaching}
\begin{tabular}{r|p{12cm}}
 \textsc{Fall 2014} & Teaching Assistant, Algorithms-II (Advanced Algorithms)\\
 &\small{Responsible for guiding 110 students in the course}\\ %% FIXME
 \multicolumn{2}{c}{} \\
 \textsc{Summer 2014} & Instructor, Advanced C\texttt{++}\\
 &\small{Taught Object Oriented Programming, Polymorphism, STL and Exceptions in C++ to a batch of more than 150 students during ACA summer school 2014}\\ %% FIXME + Course Material Link
\end{tabular}

\section{Talks}
\begin{tabular}{r|p{12cm}}
 \textsc{Fall 2014} & Unique Games Conjecture: Subhash Khot wins the Nevanlinna prize (SIGTACS Talk)\\
 &\small{The talk presented was to understand the Unique Games Conjecture (UGC) and the importance of it in Theoretical Computer Sciences. It consisted of complexity of approximating problems, PCP Theorem and the need for UGC. Finally, inapproximability results using UGC were derived and the connections between geometry, Fourier analysis and Theoretical Computer Sciences were shown.}\\
% &\small{Notes available at  \href{http://home.iitk.ac.in/~arorar/cs676/report.pdf}{home.iitk.ac.in/\textasciitilde arorar/cs676/report.pdf}}\\
 \multicolumn{2}{c}{} \\
 \textsc{Spring 2014} & A combinatorial primal–dual approach to semidefinite programs\\ 
 &\small{Primal-Dual approach has become a ubiquitous tool is deriving approximate solution using Linear Programming. This talk, motivated by the works of Sanjeev Arora, Elad Hazan, and Satyen Kale, develops a generalized primal-dual approach to solve SDPs using a modification of Matrix Multiplicative Weights update rule to symmetric matrices.}\\
 \multicolumn{2}{c}{} \\
\textsc{Fall 2013} & The Google Similarity Distance\\
&\small{This talk, motivated by the works of Rudi L. Cilibrasi, and Paul M. B. Vitanyi, presents a theory of quantifying similarity between words and phrases based on information distance and Kolmogorov Complexity. By calculating similarity based on Google page counts, applications to hierarchial clustering, classification, and language translation are discussed.}\\
\end{tabular}

\section{Selected Implementation Projects}
\begin{tabular}{r|p{12cm}}
 \textsc{Aug 2013 - Nov 2013} & Easy Cloud Storage \\
 &\small{Mentor: Prof. T.V. Prabhakar}\\
&\small{Developed a Cloud Storage Platform which integrates existing cloud services like Dropbox, Google Drive, SkyDrive and Box.net. Provided an abstract layer allowing to use these services as an extended cloud storage using a desktop application}\\
& \small{Project Link: \href{https://github.com/pankajprateek/easyCloud}{https://github.com/pankajprateek/easyCloud}}\\

 \multicolumn{2}{c}{} \\
\textsc{Spring 2013} & C++ Compiler Design\\
&\small{Implemented a working compiler for a subset of the C++ language, targeted at MIPS architecture. \textbf{Built all compiler modules}: lexer, parser with error checking grammar, symbol table and code generator; from ground up.}\\

\multicolumn{2}{c}{} \\
\textsc{Summer 2012} & Parser Generator in Python\\
&\small{Mentor: Dr. Amey Karkare}\\
& \small{Built a parser generator which performs lexical analysis on a BNF grammar and generates the corresponding parser by generating the parse tables and lexical analyser for the input grammar}\\
& \small{Project Link: \href{http://www2.cse.iitk.ac.in/karkare/parsegen/}{http://www2.cse.iitk.ac.in/karkare/parsegen/} (Internal)}\\

 \multicolumn{2}{c}{} \\
 \textsc{Jan 2012 - Apr 2012} & Using Progressive Stochastic Search to solve Sudoku CSP \\
&\small{Mentor: Prof. Amitabha Mukerjee}\\
& \small{Modelled Sudoku as a constraint satisfaction problem and implemented PSS and iterative PSS to solve a given Sudoku puzzle and observed that PSS and IPSS are better than other stochastic algorithms like Simulated Annealing and Cultural Genetic Algorithm}\\
&\small{Project Webpage: \href{http://home.iitk.ac.in/users/pratikkr/cs365/projects/}{http://home.iitk.ac.in/users/pratikkr/cs365/projects/}}\\

\end{tabular}

\section{Skills}
\begin{tabular}{rl}
Programming Languages: & Proficient (C, C++, Python, PHP), Pascal, OCaml, Assembly (x86, MIPS)\\
Utilities: & make, git, bash, \LaTeX, svn, bash, gdb, Matlab, Lex/Yacc, PLY\\
Web Technologies: & PHP, Javascript, Google App Engine, HTML/CSS\\
\end{tabular}

%Section: Scholarships and additional info
\section{Scholarships, Achievements and Awards}
\begin{tabular}{rp{13cm}}
2013 & Interned at INRIA, France through the \textbf{INRIA Internships Program}\\
2010 & \textbf{All India Rank of 157 (top 0.1\%)} in IIT-JEE exam taken by over 400,000 students\\
2010 & Gold Medal at the International Chemistry Olympiad Orientation-cum-Selection Camp at HBCSE, TIFR (36 out of 28150 candidates)\\
2009 & Olympiad Rank 7 in the International Olympiad of Mathematics\\
2008-2010 & KVPY Fellowship for budding scientists, Govt. of India (awarded to 200 students)\\
2008-2009 & National Talent Search Scholarship, Govt. of India (Merit List 24 of 1000 students)\\
2008 & Gold Medal at the State Science Talent Search Examination (SSTSE)\\
2008 & Qualified for Indian National Olympiad in Informatics (INOI)
\end{tabular}

\section{Relevant Courses}
\begin{tabular}{ll}
Approximation Algorithms & Computational Geometry\\
Semi-definite Programming & Algorithmic Information Theory\\
Artificial Intelligence Programming & Data Structures and Algorithms\\
Modern Cryptology & Data Compession\\
Algorithms-II & Data Structures and Algorithms\\
Discrete Mathematics & Compiler Design\\
Theory of Computation & Artificial Intelligence Programming
\end{tabular}

\section{Interests and Activities}
\textsc{Experience in Algorithmic and Competitive Programming}
\begin{itemize}
\item \small{Codechef (boygenius: long contest rating 1577), SPOJ (pankaj\_prateek: global rank 1007), Codeforces (boygenius: rating 1646), Hackerrank (boy\_genius: Score 2136)}
\item \small{Problem setter for various intra IITK and open-to-all contests}
\item \small{Set up judges (DomJudge) for online programming contests (ACM-ICPC style) and assignments for various courses}
\end{itemize}
\textsc{Coordinator, IOPC (International Online Programming Contest and Software Events, Techkriti 2013 (IIT Kanpur's Technical Festival)}\\
{\small Worked as a probem setter and tester for IOPC (a premier algorithmic coding contest among colleges of India). Organized 4 programming contests with \textbf{international participation}, including India's first high performance computing event on \textbf{PARAM Yuva supercomputer}, an esoteric programming language contest, and an AI battle event. The AI battle event witnessed \textbf{1100\% increase in participation} compared to the previous year.}\\\\

\section{References}
\begin{tabular}{p{5cm}|p{5cm}|p{6cm}}
\textbf{Shashank K. Mehta} & \textbf{Amitabha Mukerjee} & \textbf{Francesco Zappa Nardelli}\\
{\small Professor} & {\small Professor} & {\small Research Scientist}\\
{\small RM-503, Department of CSE} & {\small RM-507, Department of CSE} & {\small REVES, Inria Sophia-Antipolis}\\
{\small IIT Kanpur} & {\small IIT Kanpur} & {\small 2004 Route des Lucioles, BP 93}\\
{\small Kanpur-208016 India} & {\small Kanpur-208016 India} & {\small FR-06902, Sophia-Antipolis, France}\\
{\small email: \href{mailto:skmehta@cse.iitk.ac.in}{skmehta@cse.iitk.ac.in}} &
{\small email: \href{mailto:amit@cse.iitk.ac.in}{amit@cse.iitk.ac.in}} &
{\small email: \href{mailto:francesco.zappa_nardelli@inria.fr}{francesco.zappa\_nardelli@inria.fr}}
\end{tabular}\\\\


\begin{tabular}{p{5cm}p{5cm}p{6cm}}
\textbf{Amey Karkare} & &\\
{\small Assistant Professor} & &\\
{\small RM-403, Department of CSE} & &\\
{\small IIT Kanpur} & &\\
{\small Kanpur-208016 India} & &\\
{\small email: \href{mailto:karkare@cse.iitk.ac.in}{karkare@cse.iitk.ac.in}} & & \\
\end{tabular}

\end{document}
