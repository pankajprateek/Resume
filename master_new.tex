%______________________________________________________________________________________________________________________
% @brief    LaTeX2e Resume for Kamil K Wojcicki
\documentclass[margin,line]{resume}
\usepackage{hyperref}

%______________________________________________________________________________________________________________________
\begin{document}
\name{\Large Pankaj Prateek Kewalramani}
\begin{resume}

  %__________________________________________________________________________________________________________________
  % Contact Information
  \section{\mysidestyle Contact\\Information}

  Dept. of Computer Science and Engineering                            \hfill e-mail: pratikkr@cse.iitk.ac.in  \vspace{0mm}\\\vspace{0mm}%
  Indian Institute of Technology, Kanpur                         \hfill Mobile: +91 9450039421 \vspace{0mm}\       \vspace{0mm}\\\vspace{-4.5mm}%


  %__________________________________________________________________________________________________________________
  % Research Interests
  \section{\mysidestyle Research\\Interests}

  \begin{itemize}\itemsep0pt
  \item Design and Analysis of Algorithms
  \item Approximation Algorithms
  \item Competitive Programming
  \item Artificial Intelligence, Machine Learning and Natural Language Processing
  \item Number Theory and Cryptology
  \item Programming Languages
  \item Compilers and Compilers Concurrence
  \end{itemize}


  %__________________________________________________________________________________________________________________
  % Education
  \section{\mysidestyle Education}

  \textbf{Indian institute of Technology, Kanpur, India} \vspace{2mm}\\\vspace{1mm}%
  \textsl{B.Tech-M.Tech in Computer Science and Engineering} \hfill \textbf{ July 2010 -- present}\vspace{-3mm}\\\vspace{-1mm}%
  \begin{list2}
  \item Cumulative Performance Index (CPI) of \textbf{\textsf{8.9}} (on a scale of 10)
  \item Expected graduation date: July 2015
  \end{list2}\vspace{-1.5mm}
  \textbf{DAV Public School, Kota} \vspace{2mm}\\\vspace{1mm}%
  \textsl{All India Senior Secondary Certificate Examination (AISSCE)} \hfill \textbf{ July 2008 -- May 2010}\vspace{-3mm}\\\vspace{-1mm}%
  \begin{list2}
  \item Scored \textbf{\textsf{91.0\%}} marks in AISSCE Examination
  \end{list2}
  \textbf{St. Anne's Sr. Sec. School, Jodhpur} \vspace{2mm}\\\vspace{1mm}%
  \textsl{All India Secondary School Examination (AISSE)} \hfill \textbf{ July 1996 -- May 2008}\vspace{-3mm}\\\vspace{-1mm}%
  \begin{list2}
  \item Scored \textbf{\textsf {95.0\%}} marks in AISSE Examination
  \end{list2}
  \vspace{-1mm}%


  %__________________________________________________________________________________________________________________
  % Honours and Awards
  \section{\mysidestyle Scholastic Achievements}
  \begin{itemize}
  \item Secured \textbf{\textsf{All India Rank 157}} (among 4,00,000 students) in IIT-JEE 2010.
  \item Secured \textbf{\textsf{All India rank 160}} (among 11,00,000 students) in AIEEE 2010.
  \end{itemize}

  \section{\mysidestyle Internships}
  \begin{itemize}
  \item \large{\textbf{\textsf{"Partner Profile Management for Facebook Ad Exchange"}}}
    \\ \small{\textit{Summer Internship at Facebook, Menlo Park.(May 2013- July 2013)}}
    \normalsize
    \begin{itemize}
    \item Designed and Implemented a new storage framework for storing the configuration of retargeting partners
    \item Constructed a roll-out plan for moving the existing setting from legacy system to the new framework
    \item Developed an internal tool to effectively manage the configuration on the new storage framework
    \item Used XHP(a PHP extension) along with Facebook UI Libraries to build the tool
    \item Worked on features in the backend engine like throttling and targeting to improve efficiency in delivering ad impressions from the partners.
    \item Improved efficiency in delivering ad impressions from the partner by working on throttling and targeting in the
      backend engine.
    \item Modified frontend and backend clients on sync with the latest configuration changes.
    \item Incorporated Facebook technologies like generators, HipHop, storage solutions, web server stack in the config management system.
      \newpage
    \item PVF
      \begin{itemize}
      \item Worked with the Facebook Ads team, specifically on Facebook Exchange (FBX) that allows sophisticated ad partners known as Demand Side Platforms (DSPs) to programmatically buy Ad impressions on Facebook.
      \item Worked with the Facebook Ads team, specifically on Facebook Exchange (FBX) that allows sophisticated ad partners to programmatically buy ad impressions on Facebook.
      \item Worked on the Real Time Bidding (RTB) technology for Facebook Ads Team.
      \item Worked on the backend engine including some features like targeting and throttling.
      \item Designed and built an internal tool to efficiently manage the configuration of several partners.
      \item Shipped other features involving change in bid formats for the backend adexchange server. Used C++ for this.
      \item Investigated and implemented a storage layer for the config data.
      \item Worked on moving existing settings from legacy system to the new framework.
      \item Used XHP to create robust UI components and reused standard UI libraries to clean interface.
      \item Studied facebook technologies like generators, HipHop, storage solutions, web server stack, etc. and incorporated them into the tool.
      \item Added useful features like tracking changes at each partner level including ability to revert.
      \item Used Git and Mercurial for version control.
      \item Worked on both frontend and backend engineering.
      \item Modified frontend and backend clients to sync with the latest config changes.
      \item Languages Used: C++, XHP.
      \item Designed and implemented a roll-out plan for moving from old system to the new system.
      \end{itemize}
    \end{itemize}
  \end{itemize}
  %__________________________________________________________________________________________________________________

  \section{\mysidestyle Projects Undertaken}

  \begin{itemize}
  \item \large{\textbf{\textsf{"Solving and Generating Geometry Problems"}}}
    \\ \small{\textit{B.tech Project under Prof. Amitabha Mukerjee (IIT Kanpur) \& Dr. Sumit Gulwani (Microsoft Research, Redmond) \& Prof. Amey Karkare (IIT Kanpur) (July 2013 - now)}}
    \normalsize
    \begin{itemize}
    \item Aims to design an expert system that solves domain-specific geometry problems posed from the middle school textbooks
    \item Involves natural language processing of the posed question, image processing of the associated figure, conversion into intermediate representations followed by logical deductions
    \item Attempts to tutor the user by posing problems on specified concepts and of graded difficulty
    \end{itemize}

  \end{itemize}

  \section{\mysidestyle Course Projects}

  \begin{itemize}

  \item \large{\textbf{\textsf{"Easy Cloud Storage"}}}
    \\ \small{\textit{Course Project Software Engineering(August 2013 - now)}}
    \normalsize
    \begin{itemize}
    \item Developed an Easy Cloud Storage platform which integrates all existing cloud storage services such as Dropbox, Google Drive, SkyDrive etc. Provided an abstract layer to the user over his cloud accounts so that he can view them all as a single account. The system would take care of dividing the files over the cloud accounts linked.
    \item Built a web interface for authentication of the user, linking the account to other cloud storage and managing files on them.
    \item Used Software Specification Document, Architecture Diagrams and extreme programming model.
    \item Used PHP for server side programming, Twitter Bootstrap for the UI.
    \item Used API's from Dropbox, Skydrive, box.net, Google Drive.
    \item Developed a desktop application for the service with facilities like auto-syncing of files, storing some file at a particular location etc.
    \end{itemize}

  \item \large{\textbf{\textsf{"Compilers for a subset of C++ language features"}}}
    \\ \small{\textit{Course Project (Feb 2013 - April 2013)}}
    \normalsize
    \begin{itemize}
    \item Implemented a working compiler for a subset of C++ language using C++, LEX, YACC
    \item Built Lexer and error free grammar
    \item Built Parser along with semantic rules for Symbol Table, user 3-Address Code
    \item Translated the 3-Address code into Assembly language for MIPS architecture
    \end{itemize}

  \item \large{\textbf{\textsf{"Implementation of Internet Protocol Security over a chat client}}}
    \\ \small{\textit{Course Project in Computer Networks (Sept 2012- November 2012)}}
    \normalsize
    \begin{itemize}
    \item Implemented Secure IP protocols to send data packets across the network.
    \item Used Diffie-Hellman key exchange algorithm, AES encryption algorithm and authentication headers using md5 hashing
    \item Implemented Diffie-Hellman key exchange algorithm and AES encryption algorithm for the same
    \item Implemented a graphical interface to establish peer to peer connection, send and receive data
    \end{itemize}

  \item \large{\textbf{\textsf{"Extension of the Pintos Operation System"}}}
    \\ \small{\textit{Course project in Operating Systems (Sept 2012 - November 2012)}}
    \normalsize
    \begin{itemize}
    \item Worked on the pintos operating system to add features such as System Calls, shared memory between different programs, solution to the Readers-Writers problem.
    \item Implemented virtual memory with pure demand paging.
    \item Extended the pintos file-system to include sub-directories, implemented a indexed file system supporting direct, indirect and doubly indirect blocks.
    \item Added caching in the buffer during read/write operations.
    \item Includes implementation of techniques for synchronization, demand paging, scheduling, virtual memory management, page fault and swapping, and simple system calls.
    \item Implemented message queues, threads, processes, multiprogramming, scheduling policies, virtual and shared memory management and file-system
    \item Implemented and tested message queues, threads, processes, multiprogramming, memory management and file-system
    \item Complied with POSIX standards; Implemented and studied various scheduling policies viz. FCFS, SJF, RR; Used C
    \end{itemize}

  \item \large{\textbf{\textsf{"Implementation of MIPS processor on FPGA unit"}}}
    \\ \small{\textit{Course Project in Computer Organization and Architecture (August '11 - October '11)}}
    \normalsize
    \begin{itemize}
    \item Used Bluespec Verilog(BSV) to implement a simple processor on FPGA.
    \item Built the Arithmetic and Logical Unit (ALU) and a register file of 32 4-bit registers. The input and output were carried out on FPGA using buttons and 7-segment display.  
    \item Programmed a 32-bit ALU on a XILINX Spartan3 FPGA board, using BSV as hardware programming language.
    \item Arithmetic and logical operations could be easily done using this system.

    \end{itemize}

  \item \large{\textbf{\textsf{"Learning on the tron bot"}}}
    \\ \small{\textit{Summer Project under Programming Club, IIT Kanpur (May 2011 - June 2011)}}
    \normalsize
    \begin{itemize}
    \item Developed a Tron bot (based on the movie Tron), incorporating in it learning procedures so that it improves its skills as it plays against other AI bots.
    \item Built an interface using Pygame for the tron game.
    \item Built a learning model to learn by playing against human/bots. Used backtracking to adjust the weights on the feedback of the results.
    \item Tried different Backtracking approaches, and picked the best working one.
    \end{itemize}

  \item \large{\textbf{\textsf{"Simulation"}}}
    \\ \small{\textit{Electrovate, Takneek 2010, IIT Kanpur (September '10)}}
    \normalsize
    \begin{itemize}
    \item Simulated a counter which would count the number of people in a room. It would add one to the current value when a person enters a room and subtract one in the other case.
    \item This was done using LEDs, Gates, Flip-Flops, etc. that simulated a system.
    \end{itemize}

  \item \large{\textbf{\textsf{"Peer-to-peer messenger"}}}
    \\ \small{\textit{Code-a-thon 2011 under ACA, IIT Kanpur (February '11)}}
    \normalsize
    \begin{itemize}
    \item Developed a peer-to-peer messenger for exchanging messages between two computers connected on a LAN network.
    \item Used C for developing the messenger.
    \end{itemize}

  \item \large{\textbf{\textsf{"Q-Analytics"}}}
    \\ \small{\textit{Yahoo Hack U! 2011 (August '11)}}
    \normalsize
    \begin{itemize}
    \item Made a portal for online quizzing. Apart from the normal quizzing modules, it would take the question from the quizmaster and grade it for difficulty by searching for the question and the answer throughout the Web.
    \item This was done by integerating Yahoo APIs with PHP and cURL using OAuth.
    \end{itemize}

  \item \large{\textbf{\textsf{"Grounded Acquisition of Symbols"}}}
    \\ \small{\textit{Course Project in Cognitive Science (Mar '13 - Apr '13)}}
    \normalsize
    \begin{itemize}
    \item Extended the work done by Dr. Amitabha Mukherjee and Madan Dabbeeru in their paper ``Using Symbol Emergence to Discover Multi-Lingual Translations in Design''.
    \item Extended the Baby Designer Model presented in the previous paper to the native Hindi language.
    \end{itemize}

  \item \large{\textbf{\textsf{"Parser Generator in Python"}}}
    \\ \small{\textit{Mentor: Dr. Amey Karkare, CSE, IITK (May '12 - July '12)}}
    \normalsize
    \begin{itemize}
      \item Built a parser generator in Python which takes a BNF form grammar file and performs lexical analysis on it.
      \item To generate the set of rules and determine if the set is valid or not
      \item Compute the first set, follow set, LL(1) Parse Table, LR(0) Item Sets and Parse Table, SLR(1) Parse Table, LR(1) Item Sets and Parse Table and LALR(1) Item Sets and Parse Table of the given grammar file.
      \item Generated a parser to parser grammar expressions using the above.
      \item Designed the GUI using PHP and hosted the working model on the CSE local server. 
      \item Internal Link: \url{http://www2.cse.iitk.ac.in/karkare/parsegen/}
    \end{itemize}

  \item \large{\textbf{\textsf{"Using Progressive Stochastic Search to solve Sudoku Constraint Satisfaction Problem"}}}
    \\ \small{\textit{Course Project in Artificial Intelligence Programming (January '12 - April, '12)}}
    \normalsize
    \begin{itemize}
    \item Modelled Sudoku as a Constraint Satisfaction Problem
    \item Implemented the Progressive Stochastic Search and Iterative Progressive Stochastic Search algo. described in the 2003 IEEE paper ``Using PSS for solving CSPs''
    \item The related codes and documentation can be fetched from http://home.iitk.ac.in/~pratikkr/cs365/projects/
    \item  Observed that PSS and IPSS are better than other stochastic algorithms viz. Cultural Genetic Algorithm, Quantum Simulated Annealing, Repulsive Particle Swarm Optimization and Hybrid Genetic Algorithm with Simulated Annealing; Used C++
    \end{itemize}

  \item \large{\textbf{\textsf{"Automation of Post-Graduate Application System"}}}
    \\ \small{\textit{Course Project in Database Systems (January '13 - April, '13)}}
    \normalsize
    \begin{itemize}
    \item Implemented the portal having facilities for creating students profile. The profile would allow the applicant to apply online as compared to the existing system. The profiles have to be approved by an administrator.         
    \item The interface programming was done in Html and JavaScript, database built on MySql and the interaction between them was done by PHP.
    \end{itemize}

  \item \large{\textbf{\textsf{"Future TV"}}}
    \\ \small{\textit{Microsoft Code.Fun.Do '13 (18 hours)}}
    \normalsize
    \begin{itemize}
    \item Built a service on windows platform using C\# which would render personalised results for the latest and most famous videos.
    \item User can also set the genre of the videos he/she is interested in.
    \end{itemize}

  \item \large{\textbf{\textsf{"Easy Sync (Centralized Cloud Storage"}}}
    \\ \small{\textit{Microsoft Code.Fun.Do '13 (18 hours)}}
    \normalsize
    \begin{itemize}
    \item Developed an app on windows platform using C\# and Windows 8 APIs which automatically syncs all of your web hosted data storage accounts like drop box, SkyDrive account and the folders on your computer 
    \item This app helps to efficiently use the data storage space given by different platforms by removing multiple copies of files and folders stored on separate platforms.
    \item The user was provided with an abstract view of his space across various services and he could also arrange the synced files manually.
    \item Developed a proof-of-concept which could inspire furthur work.
    \item Also added with a user friendly GUI to upload or download data from the cloud.
    \item Extended this project from Windows platform to a web service and a desktop app as a part of Software Development course
    \end{itemize}

  \item \large{\textbf{\textsf{"Online Judge for Programming Contests"}}}
    \\ \small{\textit{Self-Initiative (December '12 - present)}}
    \normalsize
    \begin{itemize}
    \item Worked on the open source project, DomJudge to develop a platform for hosting online programming contests.
    \item The platform has been used to host two international contests and several intra college contests at IIT Kanpur.
    \item Leading a team of six students for further development on the project.
    \item Link to the judge: http://www2.cse.iitk.ac.in/judge/public/
    \end{itemize}
    
  \end{itemize}




  %______________________________________________________________________________________________________________________
  \section{\mysidestyle Relevant Courses}

  \begin{tabular}{@{}p{6cm}p{6.5cm}}
    - Randomized Algorithms & - Machine Learning \\
    - Introduction to Software Engineering & - Algorithms-II \\
    - Computational Complexity &- Compiler Design \\
    - Principal of Database Systems &- Introduction to Cognitive Sciences \\
    - Principles of Programming Languages &- Computer Networks \\
    - Operating Systems &- Theory of Computation \\
    - Introduction to Mathematical Logic &- Artificial Intelligence Programming \\
    - Programming Tools and Techniques &- Discrete Mathematics \\
    - Data Structures and Algorithms & - Introduction to Computer Organization \\
    - Fundamentals of Computing &- Mathematics III
  \end{tabular}


  %__________________________________________________________________________________________________________________
  % Computer Skills
  \section{\mysidestyle Technical \\Skills}

  \begin{itemize}
  \item \textbf{\textsf{Programming Languages}} - Efficient in C++, C, Python, PHP
  \item \textbf{\textsf{Other Programming Languages}} - OCaml, Pascal, Assembly Language (MIPS and x86 Architecture), BlueSpec Verilog, MySQL, Oz, Bash Script, Windows Batch Script, Basic, Foxpro
  \item \textbf{\textsf{Other Tools}} - Latex, gdb, Intel PinTool, Gnuplot, Octave, HTML5, HTML, Javascript, CSS, Xilinx ISE, AutoCAD, Lex, Yacc, Matlab, MS Office, Visual Studio, PLY (parsing tool in python)
  \item \textbf{\textsf{Hardware}} - Experience with FPGA Boards, Integerated Circuts
  \item \textbf{\textsf{Text Editor}} - Emacs, Vim
  \item \textbf{\textsf{Version Control}} - Git, Mercurial, Subversion
  \item \textbf{\textsf{Operating Systems}} - Worked on Linux(archlinux, ubuntu, linuxmint), Mac, Windows 7/8
  \end{itemize}

  %__________________________________________________________________________________________________________________
  % Computer Skills
  \section{\mysidestyle Positions of Responsibility}

  \begin{itemize}

  \item  \textbf{\textsf{Coordinator, ACA (Association of Computing Activities}}
    \begin{itemize}
      \item Mentored 3 first year students on their assigned projects given through ACA in 2012-13 to complete a project on implementation of the game Tron using Machine Learning
      \item Organising Yahoo! Hack U - a 24 hour coding marathon at IIT-K for the last 2 years in coordination with Yahoo. Received token of appreciation from Yahoo! on organising such a big event.
      \item Organized Code.Fun.Do, a 24-hour Windows App Development contest by Microsoft in 2013 and 2014.
      \item Provided semester long projects in CS field to over 50 first year students through ACA in 2012-13.
      \item Deployed an online judge (\url{http://www2.cse.iitk.ac.in:81/newonj/problem.php}) for practice of programming assignments for students for last 4 semesters with more than 750 registrants. Students from other institutes (IITB, IITR, IIITH and many more) also registered and participating.
      \item Established an online judge for algorithmic contests for 24 hour practice for ACM-ICPC and weekend contests
      \item Mentored a lecture series and training camp for ACM-ICPC for IITK students in 2012 with more than 150 students enrolled.
      \item Organised Fresher’s party for the incoming batch and Farewell for the graduating batch
    \end{itemize}

  \item  \textbf{\textsf{Coordinator, IOPC (International Online Programming Contest), Techkriti'13 August 2012-April 2013}}
    \begin{itemize}
    \item Problem Setter and Tester for IOPC.
    \item Setup the problems for IOPC, the annual 24 hr long programming contest during Techkriti (\url{www.codechef.com/IOPC2013})
    \item Worked with a team of 5 members on creating and testing the problems for the 24 hr long programming contest(\url{www.codechef.com/IOPC2013})
    \item Achieved a 100\% increase in the number of teams that participated. Total number of teams being 793.
    \end{itemize}
    
  \item  \textbf{\textsf Coordinator, Software Corner, Techkriti'13 August 2012-April 2013}
    \begin{itemize}
    \item Software Corner represents all the software related competitions during Techkriti. We conducted 5 such events.
    \item Organized the first International High Performance Computing contest, which was hosted on PARAM Yuva supercomputer in collaboration with Centre for Development of Advanced Computing (CDAC). 
    \item Problem setter for IHPC, Techkriti 2013, Asia’s first High Performance Computing programming competition
    \item Revamped Battlecity by taking the contest Online(12x participation) and getting International contestants.
    \item  Battlecity (an AI design challenge), took the contest online for the first time, got a 12x raise in the number of teams(around 600) participating in the contest.
    \item Chaos (an Unknown language programming contest), took the contest to international level and successfully managed to conduct the contests on servers from CSE department of IITK for the first time.
    \item Also conducted two onsite events Hack-a-thon and Instant (Puzzle solving and fast coding competition)
    \item Elevated three contests to international level for the first time.
    \end{itemize}

  \item \textbf{\textsf{Problem setter for Intra IITK Programming Events}}
    \begin{itemize}
    \item Organized and developed problems for many Intra IITK programming contests with more than 300 participants
    \item Lead problem setter for weekend programming contests organized by ACA and
Programming Club at IITK
    \end{itemize}

  \item \textbf{\textsf Worked as a Student Guide in the Institute Counselling Service Team 2011-2012}
    \begin{itemize}
    \item Selected through faculty interview to provide counsel and mentor 6 first year students for the term 2010-11
    \item Coordinated with a faculty-guardian to guide students and prevent ragging
    \item Helped and supervised new students during their stay in First Year.
    \item Regular meetings with them to help them adjust to the work culture of IIT Kanpur.
    \end{itemize}

  \item \textbf{\textsf Academic Mentor for the course Physics-I (Mechanics)}
    \begin{itemize}
    \item Planned and took lectures and doubt clearing sessions at hostel as well as institute level to help the students understand the fundamentals of Physics-1
    \item Mentored the students and provided them with academic assistance.
    \item Was given responsibility of two students, who were not doing well in their academics, to help them with academics and in helping them to adjust with the IIT Kanpur work culture.
    \end{itemize}
    
  %% \item \textbf{\textsf Mentored a group of 3 people to build a Intelligent twentynine card game player}
  %%   \begin{itemize}
  %%   \item Guided the group towards reading resources related to the project.
  %%   \item Provided them with constructive feedback and ideas on how to build the player
  %%   \end{itemize}

  \end{itemize}

  
  \section{\mysidestyle Other Activities}
  
  \begin{itemize}
    
  \item  \textbf{\textsf Competitive Programming}
    \begin{itemize}
    \item Won $5^{th}$ prize in Battlecity in Techkriti in 2012.
    \item An active participant of programming events in the college, was first among freshers (overall third) in the Semester Programming Contest in first year, won 1st prize in a coding contest(Kodefest) and $2^{nd}$ prize in Blackbox (puzzles and coding contest) in Takneek 2011, an inter-hostel technical event.
    \item Programming Profiles:       
      %% \begin{itemize}
      %% \item \href{http://codechef.com/users/rishabhnigam31}{Codechef} - short contest rating 82 - long contest rating 1268 in India.
      %% \item \href{http://spoj.com/users/reincarnated}{SPOJ} - global rank 2483.
      %% \item Division 1 in \href{http://codeforces.com/profile/rishabhnigam31}{Codeforces} - rating 1713 - India rank 99.
      %% \item Division 1 in \href{http://community.topcoder.com/tc?module=MemberProfile&cr=23043775}{Topcoder} - rating 1215 - India rank 222
      %% \end{itemize}
    \end{itemize}

  \end{itemize}

  \section{\mysidestyle Achievements}
  \begin{itemize}
  \item Selected for Summer Research Fellowship at Carnegie Mellon University.
  \item Selected for Summer Research Fellowship at Institut national de recherche en informatique et en automatique (INRIA), France.
  \item Ranked 12 (handle: boygenius) in the Inter-IIT Programming Contest, Interview Street : Rank list
  \item Secured All India Rank of 157 in JEE-2010 competing against 450,000 applicants
  \item Was among top 0.\% in All India Engineering Enterance Exam, 2010 conducted by CBSE (AIR-160)
  \item Awarded Gold Medal in the Orientation-cum-Selection Camp(OCSC) for the International Cheistry Olympiad conducted by HBCSE, TIFR in the year 2010
  \item Qualified for Indian National Olympiad in Informatics (INOI) in 2008
  \item Awarded the prestigious Kishore Vigyan Protsahan Yojana(KVPY) fellowship by the Department ofScience and Technology, Government of India in the year 2008
  \item Awarded the National Talent Search Examination(NTSE) Scholarship from 2008 onwards
  \item Awarded the State Talent Search Examination(SSTSE) Scholarship for securing 9th position in SSTSEorganized by the Government of Rajasthan in the year 2008
  \item Selected amongst the top \% students across the nation who appeared for the National Standard Examnation in Astronomy in the years 2009 and 2010
  \item Secured 2nd position in the national-level Map Quiz conducted by the Indian National Cartographic Association in 2007
  \end{itemize}

  %______________________________________________________________________________________________________________________
\end{resume}
\end{document}


%______________________________________________________________________________________________________________________
% EOF


